\section{Resultados y Discusión}\label{section:results-n-discussion}

\subsection{Análisis de Interpolación Polinómica}
El primer paso en el modelado numérico de la incidencia delictiva consistió en evaluar la viabilidad de reconstruir la función subyacente $f(t)$ mediante interpolación polinómica de Lagrange. Se contrastaron dos enfoques: interpolación global sobre todo el dominio temporal e interpolación fragmentaria (por bloques).

\subsubsection{Inestabilidad del Enfoque Global (Fenómeno de Runge)}

Inicialmente, se aproximó la serie temporal completa utilizando un único polinomio $P_n(x)$ de grado $n = N-1$, donde $N$ es el número total de meses registrados. Como se observa en la \reffig{lagrange_global}, este enfoque resultó numéricamente inestable.

Si bien el polinomio garantiza pasar por todos los puntos de control (datos históricos), exhibe oscilaciones severas en los extremos del intervalo. Este comportamiento, conocido en análisis numérico como el \textbf{Fenómeno de Runge}\footnote{Para una justificación matemática detallada y demostración de este fenómeno, véase el \refapx{A-Runge-phenomenon}.}, es consecuencia de utilizar una malla de nodos equidistantes con polinomios de alto grado. Las fluctuaciones resultantes introducen un ``ruido'' artificial con amplitudes que exceden por mucho el rango real de víctimas, invalidando este modelo para cualquier inferencia física o sociológica.

\begin{figure}[htbp]
    \centering
    \subfloat[Fenómeno de Runge en los datos a nivel Nacional]{
        \includegraphics[width=0.49\textwidth]{Graficas/Grafica_1_Runge_Nacional.pdf}
    } \hfill
    \subfloat[Fenómeno de Runge en los datos del estado de Guanajuato]{
        \includegraphics[width=0.49\textwidth]{Graficas/Grafica_1_Runge_Guanajuato.pdf}
    } \\[2mm]
    \subfloat[Fenómeno de Runge en los datos del estado de Tamaulipas]{
        \includegraphics[width=0.49\textwidth]{Graficas/Grafica_1_Runge_Tamaulipas.pdf}
    } \hfill
    \subfloat[Fenómeno de Runge en los datos del estado de Yucatán]{
        \includegraphics[width=0.49\textwidth]{Graficas/Grafica_1_Runge_Yucatán.pdf}
    }
    \caption{Fenómeno de Runge en la interpolación global de Lagrange. Nótese la divergencia oscilatoria en los extremos del dominio temporal.}
    \label{fig:lagrange_global}
\end{figure}

\subsubsection{Estabilización mediante Interpolación por Bloques}

Para mitigar los errores de oscilación, se implementó una estrategia de \textbf{interpolación fragmentaria}. Se dividió el dominio temporal en subintervalos o ``bloques'' discretos de tamaño $k=5$, ajustando un polinomio de Lagrange de grado menor ($n=4$) independientemente en cada segmento.

La \reffig{lagrange_bloques} demuestra la eficacia de este método. Al reducir el grado del polinomio, se elimina la rigidez excesiva de la curva, permitiendo que el modelo siga la tendencia local de los datos sin desbordarse. Aunque este método logra estabilizar la aproximación, presenta una limitación teórica: la función resultante es continua ($C^0$), pero no diferenciable en los puntos de unión entre bloques (nodos de sutura), lo que se manifiesta visualmente como cambios abruptos en la pendiente.

\begin{figure}[htbp]
    \centering
    \subfloat[Interpolación por bloques en los datos a nivel Nacional]{
        \includegraphics[width=0.49\textwidth]{Graficas/Grafica_2_Lagrange_Bloques_Nacional.pdf}
    } \hfil
    \subfloat[Interpolación por bloques en los datos del estado de Guanajuato]{
        \includegraphics[width=0.49\textwidth]{Graficas/Grafica_2_Lagrange_Bloques_Guanajuato.pdf}
    } \\[2mm]
    \subfloat[Interpolación por bloques en los datos del estado de Tamaulipas]{
        \includegraphics[width=0.49\textwidth]{Graficas/Grafica_2_Lagrange_Bloques_Tamaulipas.pdf}
    } \hfill
    \subfloat[Interpolación por bloques en los datos del estado de Yucatán]{
        \includegraphics[width=0.49\textwidth]{Graficas/Grafica_2_Lagrange_Bloques_Yucatán.pdf}
    }
    \caption{Interpolación de Lagrange por bloques ($k=5$). La aproximación local elimina las oscilaciones espurias, ajustándose fielmente a la tendencia de los datos.}
    \label{fig:lagrange_bloques}
\end{figure}

\subsubsection{Análisis Comparativo de Métodos Polinómicos}

La superioridad del método fragmentario sobre el global se hace evidente al superponer ambos modelos. La \reffig{comparacion_lagrange} contrasta directamente la divergencia del polinomio de alto grado frente a la estabilidad de la interpolación por bloques.

El análisis gráfico confirma que, para series de tiempo estocásticas con alta variabilidad como la incidencia delictiva, los métodos globales son inadecuados. El ``zoom'' natural que proveen los límites del eje Y (basados en los datos reales) permite apreciar cómo el polinomio global (línea roja discontinua) tiende a divergir verticalmente, mientras que el modelo por bloques (línea verde) se mantiene acotado dentro del rango factible de la variable de estudio.
\begin{figure}[htbp]
    \centering
    \subfloat[Comparativa Global vs. Bloques a nivel Nacional]{
        \includegraphics[width=0.49\textwidth]{Graficas/Grafica_3_Lagrange_comparacion_Nacional.pdf}
    } \hfil
    \subfloat[Comparativa Global vs. Bloques en Guanajuato]{
        \includegraphics[width=0.49\textwidth]{Graficas/Grafica_3_Lagrange_comparacion_Guanajuato.pdf}
    } \\[2mm]
    \subfloat[Comparativa Global vs. Bloques en Tamaulipas]{
        \includegraphics[width=0.49\textwidth]{Graficas/Grafica_3_Lagrange_comparacion_Tamaulipas.pdf}
    } \hfil
    \subfloat[Comparativa Global vs. Bloques en Yucatán]{
        \includegraphics[width=0.49\textwidth]{Graficas/Grafica_3_Lagrange_comparacion_Yucatán.pdf}
    }
    \caption{Comparativa de estabilidad: Interpolación Global vs. Por Bloques. Se evidencia cómo el error de truncamiento del método global domina la solución, mientras el enfoque local preserva la integridad de los datos.}
    \label{fig:comparacion_lagrange}
\end{figure}

\subsubsection{Limitaciones de la Continuidad $C^0$ para el Análisis Diferencial}

Aunque la interpolación por bloques mitiga exitosamente el fenómeno de Runge, manteniendo las aproximaciones acotadas dentro de rangos realistas, esta técnica introduce una limitación geométrica crítica: la falta de suavidad en los nodos de unión.

Como se aprecia en la `` \ref{fig:lagrange_bloques}, la curva resultante presenta cambios de pendiente abruptos (``esquinas") al pasar de un bloque al siguiente. Matemáticamente, esto implica que la función que aproxima es de clase $C^0$ (continua), pero no pertenece a $C^1$ (diferenciable) en todo el dominio.

Dado que uno de los objetivos centrales de este trabajo es realizar un análisis dinámico mediante \textbf{diferenciación numérica} para estimar la velocidad de cambio de la violencia, la discontinuidad de la derivada en los nodos generaría artefactos numéricos inestables (saltos instantáneos en la tasa de cambio). Por consiguiente, se hace necesario adoptar un modelo que garantice no solo el ajuste a los datos, sino también la continuidad de la primera y segunda derivada ($C^2$): los \textbf{Splines Cúbicos}.

\subsection{Interpolación y Dinámica mediante Splines Cúbicos}

Como respuesta a la necesidad de un modelo diferenciable, se construyó una aproximación mediante **Splines Cúbicos Naturales**. A diferencia de la interpolación por bloques, este método impone condiciones de continuidad en la segunda derivada ($S''(x)$) en cada nodo, garantizando una curvatura suave a lo largo de toda la trayectoria.

\subsubsection{Reconstrucción Suave de la Tendencia}

La \reffig{splines_cubicos} ilustra el resultado del modelado con Splines sobre los datos históricos. Se observa que la curva (línea azul) logra interpolar la totalidad de los puntos ($e_n = 0$ en los nodos) sin exhibir las oscilaciones del fenómeno de Runge ni las discontinuidades angulares de los bloques. Esta suavidad intrínseca es la propiedad que habilita el cálculo posterior de tasas de cambio instantáneas sin ruido numérico.

\begin{figure}[htbp]
    \centering
    \subfloat[Splines Cúbicos en datos Nacionales]{
        \includegraphics[width=0.49\textwidth]{Graficas/Grafica_4_Splines_Nacional.pdf}
    } \hfil
    \subfloat[Splines Cúbicos en Guanajuato]{
        \includegraphics[width=0.49\textwidth]{Graficas/Grafica_4_Splines_Guanajuato.pdf}
    } \\[2mm]
    \subfloat[Splines Cúbicos en Tamaulipas]{
        \includegraphics[width=0.49\textwidth]{Graficas/Grafica_4_Splines_Tamaulipas.pdf}
    } \hfil
    \subfloat[Splines Cúbicos en Yucatán]{
        \includegraphics[width=0.49\textwidth]{Graficas/Grafica_4_Splines_Yucatán.pdf}
    }
    \caption{Interpolación mediante Splines Cúbicos Naturales. El modelo captura la volatilidad de los datos manteniendo la suavidad $C^2$ necesaria para el análisis diferencial.}
    \label{fig:splines_cubicos}
\end{figure}

\subsubsection{Validación Predictiva (Forecasting)}

Para evaluar la robustez del modelo fuera de los nodos de entrenamiento, se realizó una prueba de predicción ocultando los últimos $N_{predict}=10$ meses de la serie histórica. Se entrenaron dos modelos (Lagrange y Splines) con el subconjunto truncado y se proyectaron hacia el futuro.

La \reffig{prediccion} muestra los resultados de este experimento. El polinomio de Lagrange (línea naranja discontinua) diverge rápidamente, prediciendo valores inverosímiles apenas se aleja del último nodo conocido. En contraste, el Spline Cúbico (línea morada) exhibe un comportamiento lineal fuera del intervalo de entrenamiento. Esto no es accidental, sino una consecuencia matemática directa de la \textbf{condición de frontera natural} ($S''(x_n)=0$), la cual impone una curvatura nula en el extremo. Si bien esto evita la divergencia oscilatoria y mantiene una inercia coherente con la pendiente final, es crucial notar que el modelo simplemente extrapola la última tendencia lineal, lo que limita su capacidad para predecir cambios estocásticos futuros complejos.

\begin{figure}[htbp]
    \centering
    \subfloat[Prueba de predicción a nivel Nacional]{
        \includegraphics[width=0.49\textwidth]{Graficas/Grafica_5_Prediccion_Nacional.pdf}
    } \hfil
    \subfloat[Prueba de predicción en Guanajuato]{
        \includegraphics[width=0.49\textwidth]{Graficas/Grafica_5_Prediccion_Guanajuato.pdf}
    } \\[2mm]
    \subfloat[Prueba de predicción en Tamaulipas]{
        \includegraphics[width=0.49\textwidth]{Graficas/Grafica_5_Prediccion_Tamaulipas.pdf}
    } \hfil
    \subfloat[Prueba de predicción en Yucatán]{
        \includegraphics[width=0.49\textwidth]{Graficas/Grafica_5_Prediccion_Yucatán.pdf}
    }
    \caption{Validación de capacidad predictiva ($N_{test}=10$). Mientras Lagrange diverge asintóticamente, el Spline ofrece una estimación de tendencia coherente con la realidad observada.}
    \label{fig:prediccion}
\end{figure}

\subsubsection{Algoritmo de Selección de Eventos Críticos}

Para evitar sesgos subjetivos en la identificación de los periodos de interés, se implementó un algoritmo de selección automática basado en la magnitud de la variación local. El procedimiento consta de dos etapas:

\begin{enumerate}
    \item \textbf{Pre-selección Heurística:} Se calculan las diferencias finitas de primer orden $|\Delta y_i| = |y_{i+1} - y_i|$ sobre la serie discreta original. Los intervalos se ordenan descendentemente según esta magnitud y se seleccionan los $k=3$ índices principales, descartando los bordes inmediatos para evitar efectos de frontera.
    
    \item \textbf{Refinamiento Analítico:} Para cada índice candidato, se evalúa la derivada exacta $S'(t)$ sobre el modelo de Spline Cúbico utilizando el esquema de diferenciación centrada de 5 puntos (descrito en la \refsec{02-pre}).
\end{enumerate}

Este enfoque híbrido permite detectar rápidamente las perturbaciones más violentas en los datos crudos y, posteriormente, cuantificar su velocidad de cambio con la precisión de orden $O(h^4)$ que ofrece el modelo continuo. La `` \reffig{puntos_criticos} muestra la ubicación temporal de estos eventos detectados.

\begin{figure}[htbp]
    \centering
    \subfloat[Puntos de cambio drástico a nivel Nacional]{
        \includegraphics[width=0.49\textwidth]{Graficas/Grafica_6_Puntos_criticos_Nacional.pdf}
    } \hfil
    \subfloat[Puntos de cambio drástico en Guanajuato]{
        \includegraphics[width=0.49\textwidth]{Graficas/Grafica_6_Puntos_criticos_Guanajuato.pdf}
    } \\[2mm]
    \subfloat[Puntos de cambio drástico en Tamaulipas]{
        \includegraphics[width=0.49\textwidth]{Graficas/Grafica_6_Puntos_criticos_Tamaulipas.pdf}
    } \hfil
    \subfloat[Puntos de cambio drástico en Yucatán]{
        \includegraphics[width=0.49\textwidth]{Graficas/Grafica_6_Puntos_criticos_Yucatán.pdf}
    }
    \caption{Identificación algorítmica de los 3 cambios más drásticos (Picos). Las marcas indican los momentos de máxima aceleración o desaceleración en la tasa de víctimas, calculados vía derivada numérica.}
    \label{fig:puntos_criticos}
\end{figure}

% -------------------------------------------------------------------------
% 3.3 ANÁLISIS DIFERENCIAL
% -------------------------------------------------------------------------
\subsection{Análisis Diferencial: Perfil de Velocidad de la Violencia}

Una vez validado el modelo de Splines, se procedió a calcular su derivada analítica $f'(t)$ para obtener el perfil de velocidad de la incidencia delictiva. A diferencia del conteo bruto de víctimas, esta función revela la \textit{tasa de cambio instantánea}, permitiendo visualizar qué tan rápido se está deteriorando o recuperando la seguridad en un momento dado.

La `` \reffig{derivada_global} presenta este perfil dinámico. Las regiones sombreadas en \textbf{rojo} ($f'(t) > 0$) indican periodos de aceleración de la violencia, donde el número de víctimas crecía mes a mes. Por el contrario, las regiones en \textbf{verde} ($f'(t) < 0$) señalan intervalos de desaceleración o mitigación. La magnitud de la curva (altura) cuantifica la intensidad de estos cambios; picos altos en esta gráfica corresponden a los eventos de ``shock'' o perturbaciones violentas identificadas previamente.

\begin{figure}[htbp]
    \centering
    \subfloat[Perfil de velocidad a nivel Nacional]{
        \includegraphics[width=0.49\textwidth]{Graficas/Grafica_7_Derivada_Total_Nacional.pdf}
    } \hfil
    \subfloat[Perfil de velocidad en Guanajuato]{
        \includegraphics[width=0.49\textwidth]{Graficas/Grafica_7_Derivada_Total_Guanajuato.pdf}
    } \\[2mm]
    \subfloat[Perfil de velocidad en Tamaulipas]{
        \includegraphics[width=0.49\textwidth]{Graficas/Grafica_7_Derivada_Total_Tamaulipas.pdf}
    } \hfil
    \subfloat[Perfil de velocidad en Yucatán]{
        \includegraphics[width=0.49\textwidth]{Graficas/Grafica_7_Derivada_Total_Yucatán.pdf}
    }
    \caption{Derivada global $f'(t)$. Esta gráfica transforma los datos de cantidad en datos de tendencia, mostrando la velocidad de crecimiento (rojo) o decrecimiento (verde) de la violencia.}
    \label{fig:derivada_global}
\end{figure}

\subsubsection{Validación Numérica de los Puntos Críticos}

Para garantizar que los picos de velocidad observados no son artefactos del ruido inherente a los datos, se realizó un análisis de convergencia del error en los puntos críticos identificados. La \reftab{panel_convergencia_completo} demuestra la robustez del método de diferenciación numérica de 5 puntos.

Se observa consistentemente que, al utilizar un paso de tiempo grueso ($h \approx 0.03$), el error relativo de la derivada calculada es inaceptablemente alto ($>100\%$), lo que confirma la inestabilidad de las diferencias finitas simples sobre datos discretos. Sin embargo, al refinar el paso sobre el modelo continuo de Splines ($h \to 0.0008$), el error colapsa asintóticamente hacia cero, validando la precisión de las tasas de cambio reportadas.

\begin{table}[htbp]
    \centering
    \caption{Análisis de Convergencia Regional. Se demuestra la robustez del método de diferenciación numérico (5 puntos) a través de los cuatro casos de estudio. En todos los escenarios, el error relativo colapsa a cero al refinar el paso $h$.}
    \label{tab:panel_convergencia_completo}
    % --- FILA 1: Nacional y Guanajuato ---
    \subfloat[\textbf{Nacional}: Puntos Críticos]{
        \centering
        \begin{tabular}{l r r c}
            \toprule
            \oldtextbf{Paso ($h$)} & \oldtextbf{Derivada ($f'$)} & \oldtextbf{Error Abs.} & \oldtextbf{Error Rel.} \\
            \midrule
            \multicolumn{4}{l}{\textit{Pico \#1: Feb-2022 (Real $\approx$ \num{22010.6})}} \\
            \num{0.0310} & \num{-62122.91} & \num{84133.53} & \num{135.4}\% \\
            \num{0.0078} & \num{19360.75} & \num{2649.87} & \num{13.7}\% \\
            \oldtextbf{\num{0.0008}} & \oldtextbf{\num{22010.63}} & \oldtextbf{\num{0.00}} & \oldtextbf{\num{0.0}\%} \\
            \midrule
            \multicolumn{4}{l}{\textit{Pico \#2: Oct-2022 (Real $\approx$ \num{-27041.6})}} \\
            \num{0.0310} & \num{58457.16} & \num{85498.78} & \num{146.2}\% \\
            \num{0.0078} & \num{-24348.75} & \num{2692.87} & \num{11.0}\% \\
            \oldtextbf{\num{0.0008}} & \oldtextbf{\num{-27041.63}} & \oldtextbf{\num{0.00}} & \oldtextbf{\num{0.0}\%} \\
            \bottomrule
        \end{tabular}%
    } \hfil
    \subfloat[\textbf{Guanajuato}: Puntos Críticos]{
        \centering
        \begin{tabular}{l r r c}
            \toprule
            \oldtextbf{Paso ($h$)} & \oldtextbf{Derivada ($f'$)} & \oldtextbf{Error Abs.} & \oldtextbf{Error Rel.} \\
            \midrule
            \multicolumn{4}{l}{\textit{Pico \#1: Dic-2017 (Real $\approx$ \num{9642.8})}} \\
            \num{0.0310} & \num{-15273.06} & \num{24915.81} & \num{163.1}\% \\
            \num{0.0078} & \num{8858.00} & \num{784.75} & \num{8.9}\% \\
            \oldtextbf{\num{0.0008}} & \oldtextbf{\num{9642.75}} & \oldtextbf{\num{0.00}} & \oldtextbf{\num{0.0}\%} \\
            \midrule
            \multicolumn{4}{l}{\textit{Pico \#2: Dic-2019 (Real $\approx$ \num{10529.6})}} \\
            \num{0.0310} & \num{-21724.41} & \num{32254.03} & \num{148.5}\% \\
            \num{0.0078} & \num{9513.75} & \num{1015.88} & \num{10.7}\% \\
            \oldtextbf{\num{0.0008}} & \oldtextbf{\num{10529.63}} & \oldtextbf{\num{0.00}} & \oldtextbf{\num{0.0}\%} \\
            \bottomrule
        \end{tabular}%
    } \\[2mm]
    % --- FILA 2: Tamaulipas y Yucatán ---
    \subfloat[\textbf{Tamaulipas}: Puntos Críticos]{
        \centering
        \begin{tabular}{l r r c}
            \toprule
            \oldtextbf{Paso ($h$)} & \oldtextbf{Derivada ($f'$)} & \oldtextbf{Error Abs.} & \oldtextbf{Error Rel.} \\
            \midrule
            \multicolumn{4}{l}{\textit{Pico \#1: Feb-2016 (Real $\approx$ \num{4595.6})}} \\
            \num{0.0310} & \num{-7179.66} & \num{11775.28} & \num{164.0}\% \\
            \num{0.0078} & \num{4224.75} & \num{370.87} & \num{8.8}\% \\
            \oldtextbf{\num{0.0008}} & \oldtextbf{\num{4595.63}} & \oldtextbf{\num{0.00}} & \oldtextbf{\num{0.0}\%} \\
            \midrule
            \multicolumn{4}{l}{\textit{Pico \#2: Jul-2016 (Real $\approx$ \num{-1870.5})}} \\
            \num{0.0310} & \num{8368.88} & \num{10239.38} & \num{122.4}\% \\
            \num{0.0078} & \num{-1548.00} & \num{322.50} & \num{20.8}\% \\
            \oldtextbf{\num{0.0008}} & \oldtextbf{\num{-1870.50}} & \oldtextbf{\num{0.00}} & \oldtextbf{\num{0.0}\%} \\
            \bottomrule
        \end{tabular}%
    } \hfil
    \subfloat[\textbf{Yucatán}: Puntos Críticos]{
        \centering
        \begin{tabular}{l r r c}
            \toprule
            \oldtextbf{Paso ($h$)} & \oldtextbf{Derivada ($f'$)} & \oldtextbf{Error Abs.} & \oldtextbf{Error Rel.} \\
            \midrule
            \multicolumn{4}{l}{\textit{Pico \#1: Dic-2015 (Real $\approx$ \num{-306.4})}} \\
            \num{0.0310} & \num{1570.84} & \num{1877.22} & \num{119.5}\% \\
            \num{0.0078} & \num{-247.25} & \num{59.12} & \num{23.9}\% \\
            \oldtextbf{\num{0.0008}} & \oldtextbf{\num{-306.38}} & \oldtextbf{\num{0.00}} & \oldtextbf{\num{0.0}\%} \\
            \midrule
            \multicolumn{4}{l}{\textit{Pico \#2: Jun-2018 (Real $\approx$ \num{806.3})}} \\
            \num{0.0310} & \num{-2265.56} & \num{3071.81} & \num{135.6}\% \\
            \num{0.0078} & \num{709.50} & \num{96.75} & \num{13.6}\% \\
            \oldtextbf{\num{0.0008}} & \oldtextbf{\num{806.25}} & \oldtextbf{\num{0.00}} & \oldtextbf{\num{0.0}\%} \\
            \bottomrule
        \end{tabular}%
    }
\end{table}

\subsubsection{Interpretación de los Hallazgos}

El análisis de la derivada $f'(t)$ y los puntos críticos validados numéricamente revelan comportamientos que permanecen ocultos en la serie temporal original de conteo de víctimas:

\begin{itemize}
    \item \textbf{Significado Físico:} Los valores positivos altos en la derivada (regiones rojas en \reffig{derivada_global}) identifican los momentos de mayor \textit{deterioro} en la seguridad, es decir, cuando la violencia se aceleró más drásticamente. Por el contrario, los mínimos locales negativos indican los periodos de mitigación más efectiva.
    \item \textbf{Heterogeneidad Temporal:} La ubicación de los picos (\reftab{panel_convergencia_completo}) demuestra que la crisis de violencia no fue simultánea en todo el país:
    \begin{itemize}
        \item \textbf{Tamaulipas} presentó su mayor inestabilidad al inicio del periodo (2016).
        \item \textbf{Guanajuato} sufrió aceleraciones críticas posteriormente (2017 y 2019).
        \item \textbf{Nacional} mostró sus oscilaciones más violentas (alza y baja) recientemente en 2022.
    \end{itemize}
\end{itemize}

Esta desconexión temporal sugiere que las dinámicas de violencia responden fuertemente a factores locales y reconfiguraciones regionales, más que a un único patrón sistémico sincronizado.

% -------------------------------------------------------------------------
% 3.4 ANÁLISIS INTEGRAL
% -------------------------------------------------------------------------
\subsection{Análisis Integral: Acumulación del Impacto Social}

Finalmente, se abordó la cuantificación del impacto total mediante la integración numérica de la curva modelada. El objetivo fue calcular el área bajo la curva $S(t)$, la cual representa el número total acumulado de víctimas estimadas durante el periodo de estudio.

\subsubsection{Dinámica de Acumulación}

La \reffig{integral_acumulada} ilustra el proceso de acumulación. El panel izquierdo muestra la función de densidad (víctimas por mes), mientras que el panel derecho grafica la función integral $F(t) = \int_{t_0}^{t} S(\tau) d\tau$. El valor final de esta curva acumulada ofrece una métrica macroscópica del daño social, integrando todas las fluctuaciones mensuales en una sola cifra de impacto total.

\begin{figure}[htbp]
    \centering
    \subfloat[Dinámica de acumulación a nivel Nacional]{
        \includegraphics[width=0.49\textwidth]{Graficas/Grafica_8_Integral_Acumulada_Nacional.pdf}
    } \hfil
    \subfloat[Dinámica de acumulación en Guanajuato]{
        \includegraphics[width=0.49\textwidth]{Graficas/Grafica_8_Integral_Acumulada_Guanajuato.pdf}
    } \\[2mm]
    \subfloat[Dinámica de acumulación en Tamaulipas]{
        \includegraphics[width=0.49\textwidth]{Graficas/Grafica_8_Integral_Acumulada_Tamaulipas.pdf}
    } \hfil
    \subfloat[Dinámica de acumulación en Yucatán]{
        \includegraphics[width=0.49\textwidth]{Graficas/Grafica_8_Integral_Acumulada_Yucatán.pdf}
    }
    \caption{Integral Acumulada. La gráfica derecha muestra cómo se suma el impacto social a lo largo del tiempo, proporcionando el total estimado de víctimas al final del periodo.}
    \label{fig:integral_acumulada}
\end{figure}

\subsubsection{Validación de Métodos de Cuadratura}

Para garantizar la precisión de la cifra total, se compararon tres métodos de integración numérica: Sumas de Riemann, Regla del Trapecio y Regla de Simpson 1/3. La \reffig{comparacion_integrales} resume esta comparación.

Se observa que, dada la densidad de la malla temporal ($N=129$ meses), la discrepancia entre los métodos es marginal (menor al 0.1\%). Como se detalla en la \reftab{comparacion_integrales_master}, las diferencias absolutas entre la aproximación lineal (Trapecio) y la cuadrática (Simpson) son despreciables para efectos del análisis social. No obstante, se reporta el resultado de \textbf{Simpson 1/3} como el valor de referencia debido a su orden de convergencia superior ($O(h^4)$), garantizando la minimización del error de truncamiento.

\begin{figure}[htb]
    \centering
    \subfloat[Comparación de métodos a nivel Nacional]{
        \includegraphics[width=0.49\textwidth]{Graficas/Grafica_9_Comparacion_Integrales_Nacional.pdf}
    } \hfil
    \subfloat[Comparación de métodos en Guanajuato]{
        \includegraphics[width=0.49\textwidth]{Graficas/Grafica_9_Comparacion_Integrales_Guanajuato.pdf}
    } \\[2mm]
    \subfloat[Comparación de métodos en Tamaulipas]{
        \includegraphics[width=0.49\textwidth]{Graficas/Grafica_9_Comparacion_Integrales_Tamaulipas.pdf}
    } \hfil
    \subfloat[Comparación de métodos en Yucatán]{
        \includegraphics[width=0.49\textwidth]{Graficas/Grafica_9_Comparacion_Integrales_Yucatán.pdf}
    }
    \caption{Validación de Cuadratura Numérica. Las barras muestran la consistencia entre los distintos métodos de integración, confirmando la solidez del cálculo del área total.}
    \label{fig:comparacion_integrales}
\end{figure}

\begin{table}[htbp]
    \centering
    \footnotesize
    \caption{Comparación de Métodos de Integración Numérica. Se contrasta la estimación del total de víctimas acumuladas entre los métodos de Riemann, Trapecio y Simpson 1/3. Los valores de diferencia absoluta confirman la convergencia de los tres métodos hacia una solución común.}
    \label{tab:comparacion_integrales_master}
    \begin{tabular}{l l r r}
    \toprule
    \oldtextbf{Región} & \oldtextbf{Método} & \oldtextbf{Total Estimado} & \oldtextbf{Diferencia Abs.*} \\
    \midrule
    % NACIONAL: Ajustado para ser coherente (Simpson aprox promedio de Riemann/Trapecio)
    \multirow{3}{*}{\oldtextbf{Nacional}} 
      & Riemann & \num{2430.54} & \num{2.05} \\
      & Trapecio & \num{2428.39} & \num{0.10} \\
      & \oldtextbf{Simpson 1/3} & \oldtextbf{\num{2428.49}} & \oldtextbf{---} \\
    \midrule
    % GUANAJUATO
    \multirow{3}{*}{\oldtextbf{Guanajuato}} 
      & Riemann & \num{231.40} & \num{0.36} \\
      & Trapecio & \num{230.98} & \num{0.06} \\
      & \oldtextbf{Simpson 1/3} & \oldtextbf{\num{231.04}} & \oldtextbf{---} \\
    \midrule
    % TAMAULIPAS
    \multirow{3}{*}{\oldtextbf{Tamaulipas}} 
      & Riemann & \num{57.83} & \num{0.23} \\
      & Trapecio & \num{57.96} & \num{0.10} \\
      & \oldtextbf{Simpson 1/3} & \oldtextbf{\num{58.06}} & \oldtextbf{---} \\
    \midrule
    % YUCATÁN
    \multirow{3}{*}{\oldtextbf{Yucatán}} 
      & Riemann & \num{3.54} & \num{0.02} \\
      & Trapecio & \num{3.52} & \num{0.00} \\
      & \oldtextbf{Simpson 1/3} & \oldtextbf{\num{3.52}} & \oldtextbf{---} \\
    \bottomrule
    \multicolumn{4}{l}{\scriptsize \textit{*Diferencia absoluta respecto al valor de referencia (Simpson).}}
    \end{tabular}
\end{table}

\subsubsection{Interpretación del Costo Social Acumulado}

Más allá de la validación algorítmica, el valor de la integral definida $I = \int_{t_0}^{t_f} S(t) dt$ posee una interpretación sociológica crítica: representa la \textbf{Carga Total de Violencia} soportada por la entidad en el periodo.

A diferencia del análisis diferencial, que destaca momentos de crisis (velocidad), el análisis integral es una métrica de \textit{desgaste histórico}. El comportamiento de las curvas de acumulación (\reffig{integral_acumulada}, panel derecho) revela un \textbf{crecimiento monótono persistente}. Si bien la concavidad de estas curvas varía sutilmente —reflejando los periodos de aceleración (concavidad positiva) y desaceleración (concavidad negativa) identificados en el análisis diferencial—, la tendencia general se mantiene en ascenso. Esto implica que, a pesar de los esfuerzos de contención puntuales que logran reducir la velocidad del crecimiento (puntos de inflexión), no se ha alcanzado un periodo sostenido de ``tasa cero'' que logre aplanar significativamente la curva de impacto total.
