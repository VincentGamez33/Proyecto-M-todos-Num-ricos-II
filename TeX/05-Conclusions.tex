\section{Conclusiones}\label{section:conclusions}

El presente estudio evidenció que la aplicación de técnicas de análisis numérico a series de tiempo sociológicas permite reconstruir funciones continuas a partir de datos discretos y extraer información dinámica oculta en los registros estadísticos. La reconstrucción de la función de incidencia delictiva mediante \textbf{Splines Cúbicos Naturales} resultó superior a la interpolación polinómica global de Lagrange para la descripción histórica, eliminando los artefactos oscilatorios del fenómeno de Runge y garantizando la suavidad ($C^2$) necesaria para el análisis diferencial.

Sin embargo, es imperativo señalar las limitaciones predictivas de estos modelos. Las pruebas de predicción (\textit{forecasting}) evidenciaron que, si bien los Splines evitan la divergencia asintótica de Lagrange, su capacidad de extrapolación es limitada. Debido a las condiciones de frontera natural, el modelo proyecta una continuación lineal de la última tendencia registrada, lo que le impide anticipar cambios estocásticos o estructurales fuera del intervalo de entrenamiento. Por consiguiente, se concluye que esta metodología constituye una herramienta robusta para el \textbf{diagnóstico ex-post}, pero debe complementarse con modelos probabilísticos para realizar proyecciones de seguridad pública a mediano plazo.

A pesar de esta limitante predictiva, la diferenciación numérica logró caracterizar exitosamente la velocidad de cambio de la violencia, revelando que las crisis de seguridad en México son fenómenos \textbf{asincrónicos}. Mientras que estados como Tamaulipas presentaron sus mayores aceleraciones al inicio del periodo (2016), otras regiones como Guanajuato y el agregado Nacional exhibieron puntos de inflexión crítica en años posteriores (2019 y 2022). Esta heterogeneidad temporal sugiere que las estrategias de mitigación deben ajustarse a los ``tiempos'' locales de cada entidad.

Finalmente, el análisis integral permitió cuantificar la \textbf{Carga Total de Violencia}. Los resultados muestran que la acumulación del daño social sigue un comportamiento monótono creciente. La coherencia numérica entre los métodos de Riemann, Trapecio y Simpson 1/3 valida esta estimación, indicando que, aunque existen periodos de desaceleración en la tasa de cambio, la inercia estructural de la violencia no ha logrado ser revertida, manteniendo el costo social acumulado en ascenso continuo, lo que evidencia la importancia de emplear modelos numéricamente robustos para el análisis retrospectivo de fenómenos complejos.