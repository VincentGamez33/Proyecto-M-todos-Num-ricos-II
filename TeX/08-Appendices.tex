\section{Justificación Metodológica de la Aproximación Polinómica por Tramos}

\subsection{Limitaciones de la Interpolación Polinómica de Alto Grado}

Para un conjunto de $n+1$ puntos de datos, como la serie de tiempo \texttt{Nacional\_Mensual} ($n=128$), es teóricamente posible construir un único polinomio de grado $n$ que pase exactamente por todos los puntos. Sin embargo, en la práctica, el uso de polinomios de interpolación de grado muy alto presenta serias desventajas numéricas.

El problema fundamental fue ilustrado por Carl Runge en 1901. El \textbf{fenómeno de Runge} demuestra que, para ciertos conjuntos de puntos (especialmente aquellos equiespaciados), el polinomio de interpolación de alto grado puede presentar \textbf{oscilaciones violentas} cerca de los extremos del intervalo. Estas oscilaciones provocan que el error de aproximación, lejos de disminuir al añadir más puntos, diverja drásticamente.

El ejemplo clásico es la función $f(x) = \frac{1}{x}$. Al interpolar esta función con un polinomio de alto grado, el polinomio resultante oscila de manera incontrolada en los extremos, fallando en representar la función real.

Los datos de este proyecto (víctimas de homicidio) son inherentemente ``ruidosos'' e irregulares. Intentar ajustar un único polinomio de grado 128 a los 129 puntos de la serie \texttt{Nacional\_Mensual} resultaría en un modelo numéricamente inestable y un ejemplo claro del fenómeno de Runge, produciendo un polinomio con oscilaciones extremas y nulo valor predictivo.

\subsection{Aproximación Polinómica por Tramos (Piecewise-Polynomial Approximation)}

Para mitigar los problemas de la interpolación de alto grado, la estrategia numéricamente robusta es la \textbf{Aproximación Polinómica por Tramos}.

Este método, descrito en textos de análisis numérico como Burden \& Faires (9ª ed., Sección 3.5), consiste en dividir el intervalo completo de datos en una serie de sub-intervalos más pequeños (o ``tramos''). En lugar de un solo polinomio de alto grado, se construye un polinomio de \textbf{grado bajo} (e.g., grado 3, 5, o 9) para cada tramo.

Las ventajas de este enfoque son significativas:
\begin{enumerate}
    \item \textbf{Evita el Fenómeno de Runge:} Al utilizar exclusivamente polinomios de grado bajo, se eliminan las oscilaciones de alto grado.
    \item \textbf{Adaptabilidad Local:} Cada polinomio se ajusta a la tendencia \textit{local} de los datos en su tramo. Esto es ideal para series de tiempo, permitiendo que el modelo se adapte a picos, valles y cambios de tendencia sin distorsionar la aproximación global.
    \item \textbf{Estabilidad Numérica:} El cálculo de varios polinomios de grado bajo es computacionalmente más simple y menos susceptible a errores de redondeo que un solo sistema de alto grado.
\end{enumerate}

\subsection{Aplicación en este Proyecto}

Este proyecto implementó una \textbf{Aproximación Polinómica por Tramos} (Sección \ref{sec:03-res}). La serie de tiempo \texttt{Nacional\_Mensual} se dividió en tramos consecutivos de 10 puntos, y para cada tramo se construyó un polinomio de interpolación de Lagrange de grado 9.

Este enfoque permite un análisis local preciso (necesario para las derivadas, Sección \ref{sec:04-conclu}) y una integración robusta (Sección \ref{sec:05-dis}), sacrificando la continuidad global de las derivadas (que no es un objetivo de este análisis) en favor de la estabilidad y la precisión local.

Una evolución de este método son los \textbf{Trazadores Cúbicos (Cubic Splines)}, los cuales son polinomios por tramos de grado 3 que imponen condiciones de continuidad en la primera y segunda derivada ($C^2$), garantizando una curva ``suave'' en los puntos de unión (nodos). Sin embargo, para los fines de este trabajo, la aproximación por tramos de Lagrange es la metodología especificada y aplicada.