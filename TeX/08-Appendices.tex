\section{Apéndice: Justificación Metodológica de la Aproximación Polinómica}\label{appendix:A-Runge-phenomenon}

\subsection{Limitaciones de la Interpolación Global (Fenómeno de Runge)}

Para un conjunto de $N$ puntos de datos, teóricamente es posible construir un único polinomio de grado $N-1$ que interpole exactamente todos los nodos. Sin embargo, en la práctica, el uso de polinomios de muy alto grado sobre mallas equidistantes presenta serias desventajas numéricas.

El problema fundamental fue descrito por Carl Runge en 1901. El \textbf{Fenómeno de Runge} demuestra que, al aumentar el grado del polinomio interpolador, el error de aproximación en los extremos del intervalo no disminuye, sino que diverge drásticamente, generando \textbf{oscilaciones violentas}.

El contraejemplo clásico es la función de Runge:
\begin{equation}
    f(x) = \frac{1}{1+25x^2}
\end{equation}
Al interpolar esta función suave con un polinomio de alto grado en nodos equiespaciados, el polinomio resultante oscila de manera incontrolada cerca de los bordes del dominio, fallando en representar la función original.

Dado que los datos de incidencia delictiva procesados en este proyecto presentan variabilidad inherente (ruido), intentar ajustar un único polinomio global (p.ej., de grado $n > 100$) resultaría en un modelo numéricamente inestable, con oscilaciones absurdas que carecen de valor predictivo o explicativo.

\subsection{Estrategia de Aproximación por Tramos}

Para mitigar este problema, la estrategia numéricamente robusta adoptada en este trabajo es la \textbf{Aproximación Polinómica por Tramos} (\textit{Piecewise-Polynomial Approximation}).

Este método, fundamentado en textos clásicos de análisis numérico (Burden \& Faires, 9ª ed., Sección 3.5), consiste en dividir el dominio temporal en subintervalos finitos. En lugar de un solo polinomio de alto grado, se construye una familia de polinomios de \textbf{grado bajo} para cada segmento.

Las ventajas de este enfoque son determinantes:
\begin{enumerate}
    \item \textbf{Eliminación del Fenómeno de Runge:} Al restringir el grado de los polinomios (e.g., cúbicos o de grado 4), se acotan las oscilaciones.
    \item \textbf{Adaptabilidad Local:} Cada polinomio responde exclusivamente a la tendencia de los datos en su vecindad, permitiendo modelar picos y valles sin distorsionar la aproximación en meses distantes.
    \item \textbf{Estabilidad Computacional:} Operar con múltiples sistemas pequeños es menos susceptible a errores de redondeo y mal condicionamiento que resolver un sistema lineal gigante de orden $N$.
\end{enumerate}

\subsection{Implementación Computacional}

Con base en lo anterior, este proyecto implementó dos niveles de aproximación por tramos para contrastar su eficacia:

\begin{itemize}
    \item \textbf{Interpolación de Lagrange por Bloques:} Se dividió la serie en bloques consecutivos de 5 puntos ($k=5$), ajustando polinomios de grado 4 en cada tramo. Esto resolvió el problema de estabilidad global, aunque introdujo puntos de no-derivabilidad ($C^0$) en las uniones.
    
    \item \textbf{Splines Cúbicos Naturales:} Como evolución del método anterior, se implementaron trazadores cúbicos (\textit{Cubic Splines}). Estos son polinomios por tramos de grado 3 que imponen condiciones de continuidad no solo en la función, sino en su primera y segunda derivada ($S \in C^2$).
\end{itemize}

Para los fines analíticos de este trabajo (cálculo de tasas de cambio e impacto acumulado), se seleccionó el modelo de \textbf{Splines Cúbicos} como el estimador óptimo, ya que garantiza la suavidad necesaria para realizar procesos de diferenciación numérica sin el ruido introducido por las discontinuidades de los bloques simples.

\section{Apéndice: Implementación Computacional}\label{appendix:codes}

A continuación se presentan los códigos fuente desarrollados en Python para la implementación de los métodos numéricos descritos en la \refsec{02-pre}.

\subsection{Interpolación}\label{apx:code-interp}

\begin{code}{Implementación de Interpolación de Lagrange}{lagrange}
    \inputminted[fontsize=\footnotesize, linenos]{python}{02-Interpolacion/interpolacion_lagrange.py}
\end{code}

\begin{code}{Implementación de Splines Cúbicos Naturales}{spline}
    \inputminted[fontsize=\footnotesize, linenos]{python}{03-Splines/interpolacion_splines.py}
\end{code}

\subsection{Cálculo Infinitesimal Numérico}\label{code-calc}

\begin{code}{Implementación de Diferenciación Numérica (Esquema de 5 puntos)}{diff}
    \inputminted[fontsize=\footnotesize, linenos]{python}{04-Diferenciacion/diferenciacion_numerica.py}
\end{code}

\begin{code}{Implementación de Integración Numérica (Riemann, Trapecio, Simpson)}{int}
    \inputminted[fontsize=\footnotesize, linenos]{python}{05-Integracion/integracion_numerica.py}
\end{code}