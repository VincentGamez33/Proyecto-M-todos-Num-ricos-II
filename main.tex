\documentclass{Disquisitio_Elementalis_Articles}

\title{Análisis de la tasa de homicidios dolosos en México utilizando Interpolación de Lagrange, Interpolación por Splines y Diferenciación e Integración Numérica}
\author{Vicente C. Gámez}
\date{October 2025}

\usepackage{lipsum}

\begin{document}

\maketitle
{Vicente Chavarría Gámez\textsuperscript{1}}
{Análisis de la tasa de homicidios dolosos en México utilizando Interpolación de Lagrange, Interpolación por Splines y Diferenciación e Integración Numérica\strut\\}
{5 de enero de 2026}
{Artículo de Investigación}
{\textsuperscript{1}\,Departamento de Matemáticas, Escuela Superior de Física y Matemáticas, Instituto Politécnico Nacional, Ciudad de México, México\strut\\
}
{Homicidio doloso, Métodos Numéricos, Series de Tiempo, Interpolación de Lagrange, Predicción (Forecasting)}
{vchavarriag1700@alumno.ipn.mx}
{%
    Este proyecto aborda el análisis numérico de una serie de datos discretos mediante técnicas de interpolación, diferenciación e integración numérica. Inicialmente se estudia la interpolación polinómica global, analizando su comportamiento y limitaciones, en particular la inestabilidad asociada al fenómeno de Runge. Como alternativa, se introduce el uso de splines cúbicos naturales, los cuales permiten construir un modelo continuo con mayor estabilidad y suavidad.
    
    A partir de dicho modelo se aplican métodos de diferenciación numérica de orden alto y reglas de cuadratura compuesta para aproximar derivadas e integrales definidas. Se presenta tanto el fundamento teórico de los métodos empleados como su implementación computacional, y se comparan los resultados obtenidos con las predicciones teóricas en términos de orden de convergencia y estabilidad. Los resultados muestran que el uso de splines cúbicos proporciona una base numéricamente robusta para el análisis de datos discretos y mejora significativamente el desempeño de los métodos de diferenciación e integración numérica.
}

\section{Introducción}\label{sec:01-intro}
\lipsum[1-6]

\begin{solucion}
    \[ e^{i\theta} \QEDwhite \]
\end{solucion}
\section{Preliminares}\label{sec:02-pre}

\subsection{Interpolación de Lagrange}
\lipsum[1]\cite{Burden}
\begin{definition}{}{lagrange_polynomials}
    Sean $x_0, x_1, \ldots, x_n$, $n + 1$ números distintos y $f$ es una función cuyos valores están dados por estos números. Entonces el único polinomio $P(x)$ de grado a lo más $n$ existe, tal que
    \[ f(x_j) = P(x_j), \quad \text{para toda } j\in\{0,1,\ldots,n\}. \]
    Dicho polinomio está dado por
    \begin{equation}
        P(x) = f(x_0)L_{n,0}(x) + f(x_1)L_{n,1}(x) + \cdots + f(x_j)L_{n,n}(x) = \sum_{j=0}^n f(x_j)L_{n,j}(x)
    \end{equation}
    donde, para toda $j \in \{0,1,\ldots,n\}$, se cumple
    \[ L_{n,j}(x) = \frac{(x-x_0)(x-x_1)\cdots(x-x_{j-1})(x-x_{j+1})\cdots(x-x_n)}{(x_j-x_0)(x_j-x_1)\cdots(x_j-x_{j-1})(x_j-x_{j+1})\cdots(x_j - x_n)}, \]
    o bien,
    \begin{equation}
        L_{n,j}(x) = \prod_{\substack{k = 0\\k \neq j}}^n \frac{(x - x_k)}{(x_j - x_k)}.
    \end{equation}
\end{definition}
% \begin{definition}{}{Lagrange_polynomials}
%     Si $x_0, x_1, \ldots, x_n$ son $n + 1$ números distintos y $f$ es una función cuyos valores están dados por estos números. Entonces el único polinomio $P(x)$ de grado a lo más $n$ existe, tal que
%     \[ f(x_k) = P(x_k), \quad \text{para cada } k\in\{0,1,\ldots,n\}. \]
%     Dicho polinomio está dado por
%     \begin{equation}
%         P(x) = f(x_0)L_{n,0}(x) + f(x_1)L_{n,1}(x) + \cdots + f(x_n)L_{n,n}(x) = \sum_{j=0}^n f(x_j)L_{n,j}(x)
%     \end{equation}
%     donde, para cada $j\in\{0,1,\ldots,n\}$, se cumple
%     \[ L_{n,j}(x) = \frac{(x-x_0)(x-x_1)\cdots(x-x_{j-1})(x-x_{j+1})\cdots(x-x_n)}{(x_j-x_0)(x_j-x_1)\cdots(x_j-x_{j-1})(x_j-x_{j+1})\cdots(x_j - x_n)}, \]
%     o bien,
%     \begin{equation}
%         L_{n,j}(x) = \prod_{\substack{k = 0\\k \neq j}}^n \frac{(x - x_k)}{(x_j - x_k)}.
%     \end{equation}
% \end{definition}

\lipsum[3-5]
\begin{code}[Implementación de los Polinomios de Lagrange en Python.]
    \inputminted{python}{Code/Python/02-Interpolacion/interpolacion_Lagrange.py}
\end{code}

\subsection{Diferenciación Numérica}
\lipsum[6-8]
\begin{code}[Implementación de la Aproximación de la Derivada en Python.]
    \inputminted{python}{Code/Python/03-Diferenciacion/derivacion_numerica.py}
\end{code}

\subsection{Integración Numérica}
\lipsum[9]
% =========================================
% Teorema para analizar el error
% =========================================
\begin{theorem}{}{TdE}
    Dada una función $f \in C^{n+1}([a,b])$ y $a \le x_0 < x_1 < x_2 < \cdots < x_n \le b$, $n+1$ números distintos. Entonces para toda $x \in [a,b]$ existe un número $\xi_x \in (a,b)$, de tal forma que
    \[ f(x) = P_n(x) + \frac{f^{n+1}(\xi_x)}{(n+1!)} \prod_{j=0}^n (x - x_j) \]
    donde $\displaystyle P_n(x) = \sum_{j=0}^n f(x_j)L_{n,j}(x)$ y $\displaystyle L_{n,j}(x) = \prod_{\substack{k = 0\\k \neq j}}^n \frac{(x - x_k)}{(x_j - x_k)}$.
\end{theorem}
% =========================================
% Teorema del Valor Medio Ponderado (con peso) para integrales
% =========================================
\begin{theorem}{}{TVMWI}
    Dada una función $f \in C([a,b])$ y dada $g:[a,b] \to \RR[+][0]$ tal que $g(x) \ge 0$ para toda $x \in [a,b]$ una función integrable. Entonces existe $c \in [a,b]$ de tal forma que
    \[ \int_a^b f(x)g(x)\,dx = f(c)\int_a^b g(x)\,dx \]
\end{theorem}
\lipsum[10]
% =========================================
% Deducción de la Regla Trapezoidal
% =========================================
\begin{theorem}{}{}
    Dada una función $f \in C^2([a,b])$. Sea $x_0 < x_1$ tal que $x_0 = a$ y $x_1 = a + h = b$ donde $h=x_1-x_0$. Demostrar que la Regla Trapezoidal está dada por
    \[ \int_{x_0}^{x_1} f(x)\,dx = \frac{h}{2}[f(x_0) + f(x_1)] - \frac{h^3\,f^{(2)}(\xi_c)}{12}. \]
\end{theorem}
\begin{demostracion}
    De acuerdo con el \refthm{TdE} y la \refdef{lagrange_polynomials} nuestra aproximación para la función $f(x)$ es de la forma
    \begin{equation}\label{eq:03-fx}
        f(x) = f(x_0)L_{1,0}(x) + f(x_1)L_{1,1}(x) + E_1(x)
    \end{equation}
    donde
    \begin{align}
        L_{1,0}(x) &= \frac{x - x_1}{x_0 - x_1} = -\frac{x - x_1}{h}, \label{eq:04-L10} \\
        L_{1,1}(x) &= \frac{x - x_0}{x_1 - x_0} = \frac{x - x_0}{h}, \label{eq:05-L11}
        \intertext{y por el \refthm{TdE}}
        E_1(x) &= \dfrac{f^{(2)}(\xi_x)}{2}(x - x_0)(x - x_1)
    \end{align}
    ---$E_1(x)$ es el término de error\footnote{Dicho termino de error será tratado más adelante}---. Luego, nuestra aproximación a la integral definida de $a$ a $b$ es de la forma
    \[ \int_a^b f(x)\,dx = \int_a^b [f(x_0)L_{1,0}(x) + f(x_1)L_{1,1}(x)]\,dx + \int_a^b E_1(x)\,dx, \]
    o bien
    \[ \int_a^b f(x)\,dx = f(x_0)\int_a^b L_{1,0}(x)\,dx + f(x_1)\int_a^b L_{1,1}(x)\,dx + \int_a^b E_1(x)\,dx \]
    Por hipótesis sabemos que $a = x_0$ y $b = x_1$, obteniendo
    \begin{equation}\label{eq:06-int}
        \int_{x_0}^{x_1} f(x)\,dx = f(x_0)\int_{x_0}^{x_1} L_{1,0}(x)\,dx + f(x_1)\int_{x_0}^{x_1} L_{1,1}(x)\,dx + \int_{x_0}^{x_1} E_1(x)\,dx.
    \end{equation}
    Ahora, sustituyendo a las \refeqn{04-L10,05-L11} e integrando por separado a $L_{1,0}(x)$ y $L_{1,1}(x)$. Se sigue para $L_{1,0}(x)$
    \[ \int_{x_0}^{x_1} L_{1,0}(x)\,dx = \int_{x_0}^{x_1}\frac{x - x_1}{h} = -\frac{1}{2h}(x - x_1)^2 \Big|_{x_0}^{x_1} = -\frac{(x_0 - x_1)^2}{2h} = \frac{(x_1 - x_0)^2}{2h} = \frac{h^2}{2h} = \frac{h}{2}. \]
    Análogamente, para $L_{1,1}(x)$ hacemos
    \[ \int_{x_0}^{x_1} L_{1,1}(x)\,dx = \int_{x_0}^{x_1} \frac{x - x_0}{h}\,dx = \frac{1}{2h} (x - x_0)^2\Big|_{x_0}^{x_1} = \frac{(x_1 - x_0)^2}{2h} = \frac{h^2}{2h} = \frac{h}{2}. \]
    Sustituyendo lo anterior en la \refeqn{06-int} tenemos
    \[ \int_{x_0}^{x_1} f(x)\,dx = f(x_0)\frac{h}{2} + f(x_1)\frac{h}{2} + \int_{x_0}^{x_1} E_1(x)\,dx, \]
    o bien
    \begin{equation}\label{eq:07-analisis-error}
        \int_{x_0}^{x_1} f(x)\,dx = \frac{h}{2}[f(x_0) + f(x_1)] + \int_{x_0}^{x_1} E_1(x)\,dx.
    \end{equation}
    Ahora, analizaremos el error. De la \refeqn{07-analisis-error} se sigue
    \[ \int_{x_0}^{x_1} E_1(x)\,dx = \int_{x_0}^{x_1} \frac{f^{(2)}(\xi_x)}{2}(x - x_0)(x - x_1)\,dx \]
    Por el \refthm{TVMWI}, sean $f(x) = \dfrac{f^{(2)}(\xi_x)}{2}$ y $g(x) = (x - x_0)(x - x_1)$. Entonces existe $c \in [a,b]$ de tal forma que $\xi_c \in (a,b)$. Esto es
    \begin{equation}\label{eq:08-ahhhh}
        \int_{x_0}^{x_1} E_1(x)\,dx = \frac{f^{(2)}(\xi_c)}{2}\int_{x_0}^{x_1} (x - x_0)(x - x_1)\,dx.
    \end{equation}
    Ahora bien, para resolver la integral, procederemos integrando por partes. Sean $u = x - x_0$ y $dv = (x - x_1)\,dx$, entonces $du = dx$ y $v = \dfrac{(x - x_1)^2}{2}$. Integrando
    \[ \int_{x_0}^{x_1} (x - x_0)(x - x_1)\,dx = (x - x_0)\frac{(x - x_1)^2}{2}\Bigg|_{x_0}^{x_1} - \frac{1}{2}\int_{x_0}^{x_1} (x - x_1)^2\,dx \]
    Al evaluar el primer término en cada límite, el resultado es 0 en ambos casos. Así,
    \[ \int_{x_0}^{x_1} (x - x_0)(x - x_1)\,dx = -\frac{1}{2 \cdot 3}(x - x_1)^3\Big|_{x_0}^{x_1} = -\frac{1}{6} \left(0 - (x_0 - x_1)^3\right) = -\frac{(x_1 - x_0)^3}{6} = -\frac{h^3}{6}. \]
    Sustituyendo el valor de la integral en la \refeqn{08-ahhhh}, obtenemos
    \[ \int_{x_0}^{x_1} E_1(x)\,dx = -\frac{f^{(2)}(\xi_c)}{2}\cdot\frac{h^3}{6}, \]
    o bien,
    \begin{equation}\label{eq:09-ahhhhhhhhh}
        \int_{x_0}^{x_1} E_1(x)\,dx = -\frac{h^3\,f^{(2)}(\xi_c)}{12}.
    \end{equation}
    Finalmente, sustituyendo la \refeqn{09-ahhhhhhhhh} en la \refeqn{07-analisis-error}
    \begin{equation}
        \int_{x_0}^{x_1} f(x)\,dx = \frac{h}{2}[f(x_0) + f(x_1)] - \frac{h^3\,f^{(2)}(\xi_c)}{12}.
    \end{equation}
    Por lo tanto, queda demostrado que la \textbf{Regla Trapezoidal} es de la forma
    \[ \int_{x_0}^{x_1} f(x)\,dx = \frac{h}{2}[f(x_0) + f(x_1)] - \frac{h^3\,f^{(2)}(\xi_c)}{12}. \QEDblack \]
\end{demostracion}
\begin{code}[Implementación de la Aproximación de la Integral de Riemann en Python.]
    \inputminted{python}{Code/Python/04-Integracion/integracion_numerica.py}
\end{code}
\section{Resultados}\label{sec:03-res}

\lipsum[1-2]
\begin{figure}[H]
    \centering
    \includegraphics[width=0.5\linewidth]{Code/Python/01-Limpieza Datos/01-a-Nacional/homicidios_nacional.png}
    \caption{Gráfica de los homicidios dolosos en México desde 2015 hasta 2025.}
    \label{fig:Nacional-Anual}
\end{figure}

\lipsum[3-4]
\begin{figure}[H]
    \centering
    \subfloat[Guanajuato tiene el mayor índice de homicidios dolosos.]{
        \includegraphics[width=0.48\textwidth]{Code/Python/01-Limpieza Datos/01-b-Estatal/grafica_Mayor_Guanajuato.png}
    } \hfill
    \subfloat[]{
        \includegraphics[width=0.48\textwidth]{Code/Python/01-Limpieza Datos/01-b-Estatal/grafica_Medio_Zacatecas.png}
    } \\
    \subfloat[]{
        \includegraphics[width=0.48\textwidth]{Code/Python/01-Limpieza Datos/01-b-Estatal/grafica_Menor_Yucatán.png}
    }
    \caption{}
    \label{fig:Estados-Mayor-Menor-Promedio}
\end{figure}

\newpage
% \section{Discusión}\label{section:05-dis}
\section{Conclusiones}\label{section:conclusions}

El presente estudio evidenció que la aplicación de técnicas de análisis numérico a series de tiempo sociológicas permite reconstruir funciones continuas a partir de datos discretos y extraer información dinámica oculta en los registros estadísticos. La reconstrucción de la función de incidencia delictiva mediante \textbf{Splines Cúbicos Naturales} resultó superior a la interpolación polinómica global de Lagrange para la descripción histórica, eliminando los artefactos oscilatorios del fenómeno de Runge y garantizando la suavidad ($C^2$) necesaria para el análisis diferencial.

Sin embargo, es imperativo señalar las limitaciones predictivas de estos modelos. Las pruebas de predicción (\textit{forecasting}) evidenciaron que, si bien los Splines evitan la divergencia asintótica de Lagrange, su capacidad de extrapolación es limitada. Debido a las condiciones de frontera natural, el modelo proyecta una continuación lineal de la última tendencia registrada, lo que le impide anticipar cambios estocásticos o estructurales fuera del intervalo de entrenamiento. Por consiguiente, se concluye que esta metodología constituye una herramienta robusta para el \textbf{diagnóstico ex-post}, pero debe complementarse con modelos probabilísticos para realizar proyecciones de seguridad pública a mediano plazo.

A pesar de esta limitante predictiva, la diferenciación numérica logró caracterizar exitosamente la velocidad de cambio de la violencia, revelando que las crisis de seguridad en México son fenómenos \textbf{asincrónicos}. Mientras que estados como Tamaulipas presentaron sus mayores aceleraciones al inicio del periodo (2016), otras regiones como Guanajuato y el agregado Nacional exhibieron puntos de inflexión crítica en años posteriores (2019 y 2022). Esta heterogeneidad temporal sugiere que las estrategias de mitigación deben ajustarse a los ``tiempos'' locales de cada entidad.

Finalmente, el análisis integral permitió cuantificar la \textbf{Carga Total de Violencia}. Los resultados muestran que la acumulación del daño social sigue un comportamiento monótono creciente. La coherencia numérica entre los métodos de Riemann, Trapecio y Simpson 1/3 valida esta estimación, indicando que, aunque existen periodos de desaceleración en la tasa de cambio, la inercia estructural de la violencia no ha logrado ser revertida, manteniendo el costo social acumulado en ascenso continuo, lo que evidencia la importancia de emplear modelos numéricamente robustos para el análisis retrospectivo de fenómenos complejos.
\section{Agradecimientos}\label{sec:06-agra}
\input{TeX/07-Bibliographic_References}

\appendix
\section{Justificación Metodológica de la Aproximación Polinómica por Tramos}

\subsection{Limitaciones de la Interpolación Polinómica de Alto Grado}

Para un conjunto de $n+1$ puntos de datos, como la serie de tiempo \texttt{Nacional\_Mensual} ($n=128$), es teóricamente posible construir un único polinomio de grado $n$ que pase exactamente por todos los puntos. Sin embargo, en la práctica, el uso de polinomios de interpolación de grado muy alto presenta serias desventajas numéricas.

El problema fundamental fue ilustrado por Carl Runge en 1901. El \textbf{fenómeno de Runge} demuestra que, para ciertos conjuntos de puntos (especialmente aquellos equiespaciados), el polinomio de interpolación de alto grado puede presentar \textbf{oscilaciones violentas} cerca de los extremos del intervalo. Estas oscilaciones provocan que el error de aproximación, lejos de disminuir al añadir más puntos, diverja drásticamente.

El ejemplo clásico es la función $f(x) = \frac{1}{x}$. Al interpolar esta función con un polinomio de alto grado, el polinomio resultante oscila de manera incontrolada en los extremos, fallando en representar la función real.

Los datos de este proyecto (víctimas de homicidio) son inherentemente ``ruidosos'' e irregulares. Intentar ajustar un único polinomio de grado 128 a los 129 puntos de la serie \texttt{Nacional\_Mensual} resultaría en un modelo numéricamente inestable y un ejemplo claro del fenómeno de Runge, produciendo un polinomio con oscilaciones extremas y nulo valor predictivo.

\subsection{Aproximación Polinómica por Tramos (Piecewise-Polynomial Approximation)}

Para mitigar los problemas de la interpolación de alto grado, la estrategia numéricamente robusta es la \textbf{Aproximación Polinómica por Tramos}.

Este método, descrito en textos de análisis numérico como Burden \& Faires (9ª ed., Sección 3.5), consiste en dividir el intervalo completo de datos en una serie de sub-intervalos más pequeños (o ``tramos''). En lugar de un solo polinomio de alto grado, se construye un polinomio de \textbf{grado bajo} (e.g., grado 3, 5, o 9) para cada tramo.

Las ventajas de este enfoque son significativas:
\begin{enumerate}
    \item \textbf{Evita el Fenómeno de Runge:} Al utilizar exclusivamente polinomios de grado bajo, se eliminan las oscilaciones de alto grado.
    \item \textbf{Adaptabilidad Local:} Cada polinomio se ajusta a la tendencia \textit{local} de los datos en su tramo. Esto es ideal para series de tiempo, permitiendo que el modelo se adapte a picos, valles y cambios de tendencia sin distorsionar la aproximación global.
    \item \textbf{Estabilidad Numérica:} El cálculo de varios polinomios de grado bajo es computacionalmente más simple y menos susceptible a errores de redondeo que un solo sistema de alto grado.
\end{enumerate}

\subsection{Aplicación en este Proyecto}

Este proyecto implementó una \textbf{Aproximación Polinómica por Tramos} (Sección \ref{sec:03-res}). La serie de tiempo \texttt{Nacional\_Mensual} se dividió en tramos consecutivos de 10 puntos, y para cada tramo se construyó un polinomio de interpolación de Lagrange de grado 9.

Este enfoque permite un análisis local preciso (necesario para las derivadas, Sección \ref{sec:04-conclu}) y una integración robusta (Sección \ref{sec:05-dis}), sacrificando la continuidad global de las derivadas (que no es un objetivo de este análisis) en favor de la estabilidad y la precisión local.

Una evolución de este método son los \textbf{Trazadores Cúbicos (Cubic Splines)}, los cuales son polinomios por tramos de grado 3 que imponen condiciones de continuidad en la primera y segunda derivada ($C^2$), garantizando una curva ``suave'' en los puntos de unión (nodos). Sin embargo, para los fines de este trabajo, la aproximación por tramos de Lagrange es la metodología especificada y aplicada.

\end{document}
